%!TEX encoding = IsoLatin
\newcommand{\MYLANGUAGE}{english}
\newcommand{\MYMAINFONT}{Georgia}
\newcommand{\MYSANSFONT}{Georgia}

\documentclass[
a4paper,
11pt,
%appendixprefix,
%openany,
%nochapterprefix,
%normalheadings,
parskip,
\MYLANGUAGE,
abstractoff,
bibliography=totoc
]{scrartcl}
% scrbook

\XeTeXinputencoding latin1 


\newcommand{\TITLEONE}{Case M}
\newcommand{\TITLESUB}{An application of the ODESSA method}
%\newcommand{\DATE}{January 20, 2012}
\newcommand{\AUTHOR}{Martin von Weissenberg and Damian Andrew Tamburri}
%\newcommand{\KEYWORDS}{Agile, enterprise, organization}

\newtheorem{hypo}{Hypothesis}

%\usepackage[latin1]{inputenc}
\usepackage{url}
%\usepackage{setspace}
\usepackage{natbib}
\usepackage[right]{eurosym}
\usepackage{appendix}
\usepackage{colortbl}

\usepackage{mdwlist}

\usepackage[
pdfauthor={\AUTHOR},
pdftitle={\TITLEONE --- \TITLESUB},
%pdfkeywords={\KEYWORDS},
pdfpagemode=UseOutlines,
bookmarksnumbered=true,
breaklinks
]{hyperref}




%% XeLaTeX
\usepackage{polyglossia}\setdefaultlanguage{\MYLANGUAGE}
\usepackage{xltxtra}
\setmainfont[Mapping=tex-text]{\MYMAINFONT}
\setsansfont[Mapping=tex-text]{\MYSANSFONT}
\areaset{150mm}{237mm}

%% PDFLaTeX

%\usepackage{times}\usepackage[T1]{fontenc} % Times
% The default is Computer Modern (T1)

\usepackage{setspace}
%\singlespacing
\onehalfspacing
%\doublespacing
%\setstretch{1.1}


\pagestyle{plain}
\bibliographystyle{abbrvnat} % could also be plainnat or alpha


\newif\ifpdf
\ifx\pdfoutput\undefined
\pdffalse % we are not running PDFLaTeX
\else
\pdfoutput=1 % we are running PDFLaTeX
\pdftrue
\fi

\ifpdf
\usepackage[pdftex]{graphicx}
\DeclareGraphicsExtensions{.pdf, .jpg, .png}
\else
\usepackage{graphicx}
\DeclareGraphicsExtensions{.eps, .jpg, .png}
\fi

\bibpunct{(}{)}{;}{a}{,}{,}


\begin{document}

%\makecompactlist{itemize*}{itemize}
%\makecompactlist{enumerate*}{enumerate}
%\makecompactlist{description*}{description}


\title{\TITLEONE}
\@ifundefined{TITLESUB}{ }{\subject{\TITLESUB}}
\author{\AUTHOR}
\@ifundefined{DATE}{ }{\date{\DATE}}

\setlength{\parindent}{0pt}
\setlength{\parskip}{12pt plus 6pt minus 3pt}

\setlength{\textfloatsep}{36pt plus 12.0pt minus 8.0pt} % top or bottom figures [tb]
\setlength{\floatsep}{20.0pt plus 4.0pt minus 2.0pt} % spacing between figures
\setlength{\intextsep}{20.0pt plus 4.0pt minus 2.0pt} % embedded figures [h]

\ifpdf
\DeclareGraphicsExtensions{.pdf, .jpg, .tif}
\else
\DeclareGraphicsExtensions{.eps, .jpg, .pdf}
\fi

\maketitle

%\begin{abstract}
%Abstract
%\end{abstract}

\setcounter{tocdepth}{2}
\tableofcontents

\section{Introduction}

M was an R{\&}D department at N, focused on developing a new mobile software platform and a series of devices using that platform. It was formed in early 2008 from a previous department of 180 employees and some 50--60 employees in additional teams. The M department grew on average 30\% per quarter for ten quarters straight and had over 2200 employees by the end of 2010. In February 2011, the new CEO of N announced a change of technology strategy, and the M endeavor was ramped down over the following 18 months.

We've chosen to study the M organization at its largest, around the end of 2010. By this time, the organization had grown to more than 2200 employees in approx.~150 teams, of which more than 100 were software development teams.


\section{Background}


\section{Methodology}


\section{Analysis}

\subsection{Macro-scale}


\begin{table}[htbp]
\begin{center}
%\vspace{2em}
\begin{tabular}{lp{100mm}}
STRUCTURE & {\bf Yes.} The organization was large and multi-layered.\\
SITUATEDNESS & {\bf No.} \\
DISPERSION & {\bf Yes.} The organization was dispersed over dozens of sites, with at least five major sites in three timezones.\\
INFORMALITY & {\bf No.} There were lots of organization-wide intranet tools that formalized the interaction around certain topics like product management and defect tracking.\\
DURATION & {\bf No.} There were ''support teams'' for common tasks such as quality management, testing and build infrastructure.\\
VISIBILITY & {\bf No.} The organization was actively working with open source and was therefore nominally in favor of sharing information, but in practice management was based on the withholding of information. \\
COHESION & {\bf No.} Aside from the corporate tools, each workgroup had their own specific methods. \\
ROI & {\bf No.} This R{\&}D organization was not expected to directly generate revenue. \\
ENGAGEMENT & {\bf ?} \\
CULTURE & {\bf ?} \\
FORMALITY & {\bf ?} \\
PROBLEM & {\bf ?} \\
GOVERNANCE & {\bf ?} \\
\end{tabular}
\vspace{1em}
\caption{Macro-scale analysis of M}
\label{tab:macro1}
\end{center}
\end{table}






\subsection{Micro-scale}

\subsubsection{The Connectivity Team}

The connectivity team consisted of 15--30 team members, including a manager, two product owners, and two or three testers. The Connectivity team contributed heavily to open source software components, including the connection manager ConnMan\footnote{See \url{http://connman.net/}.} and the bluetooth stack BlueZ\footnote{See \url{http://www.bluez.org/}.}. They were also instrumental in carrying out various radio certifications for the M platform and products.

\begin{table}[htbp]
\begin{center}
%\vspace{2em}
\begin{tabular}{lp{100mm}}
STRUCTURE & {\bf ?} \\
SITUATEDNESS & {\bf ?} \\
DISPERSION & {\bf ?} \\
INFORMALITY & {\bf ?} \\
DURATION & {\bf ?} \\
VISIBILITY & {\bf ?} \\
COHESION & {\bf ?} \\
ROI & {\bf ?} \\
ENGAGEMENT & {\bf ?} \\
CULTURE & {\bf ?} \\
FORMALITY & {\bf ?} \\
PROBLEM & {\bf ?} \\
GOVERNANCE & {\bf ?} \\
\end{tabular}
\vspace{1em}
\caption{Micro-scale analysis of M Connectivity}
\label{tab:microconn}
\end{center}
\end{table}

\subsubsection{The Multimedia Team}

The multimedia team worked on the multimedia software stack, including audio and video frameworks and media players. Components included GStreamer\footnote{See \url{http://gstreamer.freedesktop.org/}.}.

\begin{table}[htbp]
\begin{center}
%\vspace{2em}
\begin{tabular}{lp{100mm}}
STRUCTURE & {\bf ?} \\
SITUATEDNESS & {\bf ?} \\
DISPERSION & {\bf ?} \\
INFORMALITY & {\bf ?} \\
DURATION & {\bf ?} \\
VISIBILITY & {\bf ?} \\
COHESION & {\bf ?} \\
ROI & {\bf ?} \\
ENGAGEMENT & {\bf ?} \\
CULTURE & {\bf ?} \\
FORMALITY & {\bf ?} \\
PROBLEM & {\bf ?} \\
GOVERNANCE & {\bf ?} \\
\end{tabular}
\vspace{1em}
\caption{Micro-scale analysis of M Multimedia}
\label{tab:microconn}
\end{center}
\end{table}



\section{Results}


\section{Conclusions}


%\addcontentsline{toc}{section}{Bibliography}
\bibliography{/Users/mvonweis/Documents/Agile/AgileAtScale/book/book}


%\appendix
%\appendixpage




\end{document}
