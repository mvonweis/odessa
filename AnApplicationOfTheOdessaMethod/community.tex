%% bare_conf.tex
%% V1.3
%% 2007/01/11
%% by Michael Shell
%% See:
%% http://www.michaelshell.org/
%% for current contact information.
%%
%% This is a skeleton file demonstrating the use of IEEEtran.cls
%% (requires IEEEtran.cls version 1.7 or later) with an IEEE conference paper.
%%
%% Support sites:
%% http://www.michaelshell.org/tex/ieeetran/
%% http://www.ctan.org/tex-archive/macros/latex/contrib/IEEEtran/
%% and
%% http://www.ieee.org/

%%*************************************************************************
%% Legal Notice:
%% This code is offered as-is without any warranty either expressed or
%% implied; without even the implied warranty of MERCHANTABILITY or
%% FITNESS FOR A PARTICULAR PURPOSE! 
%% User assumes all risk.
%% In no event shall IEEE or any contributor to this code be liable for
%% any damages or losses, including, but not limited to, incidental,
%% consequential, or any other damages, resulting from the use or misuse
%% of any information contained here.
%%
%% All comments are the opinions of their respective authors and are not
%% necessarily endorsed by the IEEE.
%%
%% This work is distributed under the LaTeX Project Public License (LPPL)
%% ( http://www.latex-project.org/ ) version 1.3, and may be freely used,
%% distributed and modified. A copy of the LPPL, version 1.3, is included
%% in the base LaTeX documentation of all distributions of LaTeX released
%% 2003/12/01 or later.
%% Retain all contribution notices and credits.
%% ** Modified files should be clearly indicated as such, including  **
%% ** renaming them and changing author support contact information. **
%%
%% File list of work: IEEEtran.cls, IEEEtran_HOWTO.pdf, bare_adv.tex,
%%                    bare_conf.tex, bare_jrnl.tex, bare_jrnl_compsoc.tex
%%*************************************************************************

% *** Authors should verify (and, if needed, correct) their LaTeX system  ***
% *** with the testflow diagnostic prior to trusting their LaTeX platform ***
% *** with production work. IEEE's font choices can trigger bugs that do  ***
% *** not appear when using other class files.                            ***
% The testflow support page is at:
% http://www.michaelshell.org/tex/testflow/



% Note that the a4paper option is mainly intended so that authors in
% countries using A4 can easily print to A4 and see how their papers will
% look in print - the typesetting of the document will not typically be
% affected with changes in paper size (but the bottom and side margins will).
% Use the testflow package mentioned above to verify correct handling of
% both paper sizes by the user's LaTeX system.
%
% Also note that the "draftcls" or "draftclsnofoot", not "draft", option
% should be used if it is desired that the figures are to be displayed in
% draft mode.
%
\documentclass[10pt,journal,letterpaper,compsoc]{IEEEtran}
% Add the compsoc option for Computer Society conferences.
%
% If IEEEtran.cls has not been installed into the LaTeX system files,
% manually specify the path to it like:
% \documentclass[conference]{../sty/IEEEtran}





% Some very useful LaTeX packages include:
% (uncomment the ones you want to load)


% *** MISC UTILITY PACKAGES ***
%
%\usepackage{ifpdf}
% Heiko Oberdiek's ifpdf.sty is very useful if you need conditional
% compilation based on whether the output is pdf or dvi.
% usage:
% \ifpdf
%   % pdf code
% \else
%   % dvi code
% \fi
% The latest version of ifpdf.sty can be obtained from:
% http://www.ctan.org/tex-archive/macros/latex/contrib/oberdiek/
% Also, note that IEEEtran.cls V1.7 and later provides a builtin
% \ifCLASSINFOpdf conditional that works the same way.
% When switching from latex to pdflatex and vice-versa, the compiler may
% have to be run twice to clear warning/error messages.






% *** CITATION PACKAGES ***
%
%\usepackage{cite}
% cite.sty was written by Donald Arseneau
% V1.6 and later of IEEEtran pre-defines the format of the cite.sty package
% \cite{} output to follow that of IEEE. Loading the cite package will
% result in citation numbers being automatically sorted and properly
% "compressed/ranged". e.g., [1], [9], [2], [7], [5], [6] without using
% cite.sty will become [1], [2], [5]--[7], [9] using cite.sty. cite.sty's
% \cite will automatically add leading space, if needed. Use cite.sty's
% noadjust option (cite.sty V3.8 and later) if you want to turn this off.
% cite.sty is already installed on most LaTeX systems. Be sure and use
% version 4.0 (2003-05-27) and later if using hyperref.sty. cite.sty does
% not currently provide for hyperlinked citations.
% The latest version can be obtained at:
% http://www.ctan.org/tex-archive/macros/latex/contrib/cite/
% The documentation is contained in the cite.sty file itself.






% *** GRAPHICS RELATED PACKAGES ***
%
\ifCLASSINFOpdf
   \usepackage[pdftex]{graphicx}
  % declare the path(s) where your graphic files are
  % \graphicspath{{../pdf/}{../jpeg/}}
  % and their extensions so you won't have to specify these with
  % every instance of \includegraphics
  % \DeclareGraphicsExtensions{.pdf,.jpeg,.png}
\else
  % or other class option (dvipsone, dvipdf, if not using dvips). graphicx
  % will default to the driver specified in the system graphics.cfg if no
  % driver is specified.
   \usepackage[dvips]{graphicx}
  % declare the path(s) where your graphic files are
  % \graphicspath{{../eps/}}
  % and their extensions so you won't have to specify these with
  % every instance of \includegraphics
  % \DeclareGraphicsExtensions{.eps}
\fi
% graphicx was written by David Carlisle and Sebastian Rahtz. It is
% required if you want graphics, photos, etc. graphicx.sty is already
% installed on most LaTeX systems. The latest version and documentation can
% be obtained at: 
% http://www.ctan.org/tex-archive/macros/latex/required/graphics/
% Another good source of documentation is "Using Imported Graphics in
% LaTeX2e" by Keith Reckdahl which can be found as epslatex.ps or
% epslatex.pdf at: http://www.ctan.org/tex-archive/info/
%
% latex, and pdflatex in dvi mode, support graphics in encapsulated
% postscript (.eps) format. pdflatex in pdf mode supports graphics
% in .pdf, .jpeg, .png and .mps (metapost) formats. Users should ensure
% that all non-photo figures use a vector format (.eps, .pdf, .mps) and
% not a bitmapped formats (.jpeg, .png). IEEE frowns on bitmapped formats
% which can result in "jaggedy"/blurry rendering of lines and letters as
% well as large increases in file sizes.
%
% You can find documentation about the pdfTeX application at:
% http://www.tug.org/applications/pdftex





% *** MATH PACKAGES ***
%
%\usepackage[cmex10]{amsmath}
% A popular package from the American Mathematical Society that provides
% many useful and powerful commands for dealing with mathematics. If using
% it, be sure to load this package with the cmex10 option to ensure that
% only type 1 fonts will utilized at all point sizes. Without this option,
% it is possible that some math symbols, particularly those within
% footnotes, will be rendered in bitmap form which will result in a
% document that can not be IEEE Xplore compliant!
%
% Also, note that the amsmath package sets \interdisplaylinepenalty to 10000
% thus preventing page breaks from occurring within multiline equations. Use:
%\interdisplaylinepenalty=2500
% after loading amsmath to restore such page breaks as IEEEtran.cls normally
% does. amsmath.sty is already installed on most LaTeX systems. The latest
% version and documentation can be obtained at:
% http://www.ctan.org/tex-archive/macros/latex/required/amslatex/math/





% *** SPECIALIZED LIST PACKAGES ***
%
%\usepackage{algorithmic}
% algorithmic.sty was written by Peter Williams and Rogerio Brito.
% This package provides an algorithmic environment fo describing algorithms.
% You can use the algorithmic environment in-text or within a figure
% environment to provide for a floating algorithm. Do NOT use the algorithm
% floating environment provided by algorithm.sty (by the same authors) or
% algorithm2e.sty (by Christophe Fiorio) as IEEE does not use dedicated
% algorithm float types and packages that provide these will not provide
% correct IEEE style captions. The latest version and documentation of
% algorithmic.sty can be obtained at:
% http://www.ctan.org/tex-archive/macros/latex/contrib/algorithms/
% There is also a support site at:
% http://algorithms.berlios.de/index.html
% Also of interest may be the (relatively newer and more customizable)
% algorithmicx.sty package by Szasz Janos:
% http://www.ctan.org/tex-archive/macros/latex/contrib/algorithmicx/




% *** ALIGNMENT PACKAGES ***
%
%\usepackage{array}
% Frank Mittelbach's and David Carlisle's array.sty patches and improves
% the standard LaTeX2e array and tabular environments to provide better
% appearance and additional user controls. As the default LaTeX2e table
% generation code is lacking to the point of almost being broken with
% respect to the quality of the end results, all users are strongly
% advised to use an enhanced (at the very least that provided by array.sty)
% set of table tools. array.sty is already installed on most systems. The
% latest version and documentation can be obtained at:
% http://www.ctan.org/tex-archive/macros/latex/required/tools/


%\usepackage{mdwmath}
%\usepackage{mdwtab}
% Also highly recommended is Mark Wooding's extremely powerful MDW tools,
% especially mdwmath.sty and mdwtab.sty which are used to format equations
% and tables, respectively. The MDWtools set is already installed on most
% LaTeX systems. The lastest version and documentation is available at:
% http://www.ctan.org/tex-archive/macros/latex/contrib/mdwtools/


% IEEEtran contains the IEEEeqnarray family of commands that can be used to
% generate multiline equations as well as matrices, tables, etc., of high
% quality.


%\usepackage{eqparbox}
% Also of notable interest is Scott Pakin's eqparbox package for creating
% (automatically sized) equal width boxes - aka "natural width parboxes".
% Available at:
% http://www.ctan.org/tex-archive/macros/latex/contrib/eqparbox/


\usepackage{array}


% *** SUBFIGURE PACKAGES ***
%\usepackage[tight,footnotesize]{subfigure}
% subfigure.sty was written by Steven Douglas Cochran. This package makes it
% easy to put subfigures in your figures. e.g., "Figure 1a and 1b". For IEEE
% work, it is a good idea to load it with the tight package option to reduce
% the amount of white space around the subfigures. subfigure.sty is already
% installed on most LaTeX systems. The latest version and documentation can
% be obtained at:
% http://www.ctan.org/tex-archive/obsolete/macros/latex/contrib/subfigure/
% subfigure.sty has been superceeded by subfig.sty.



%\usepackage[caption=false]{caption}
%\usepackage[font=footnotesize]{subfig}
% subfig.sty, also written by Steven Douglas Cochran, is the modern
% replacement for subfigure.sty. However, subfig.sty requires and
% automatically loads Axel Sommerfeldt's caption.sty which will override
% IEEEtran.cls handling of captions and this will result in nonIEEE style
% figure/table captions. To prevent this problem, be sure and preload
% caption.sty with its "caption=false" package option. This is will preserve
% IEEEtran.cls handing of captions. Version 1.3 (2005/06/28) and later 
% (recommended due to many improvements over 1.2) of subfig.sty supports
% the caption=false option directly:
%\usepackage[caption=false,font=footnotesize]{subfig}
%
% The latest version and documentation can be obtained at:
% http://www.ctan.org/tex-archive/macros/latex/contrib/subfig/
% The latest version and documentation of caption.sty can be obtained at:
% http://www.ctan.org/tex-archive/macros/latex/contrib/caption/




% *** FLOAT PACKAGES ***
%
%\usepackage{fixltx2e}
% fixltx2e, the successor to the earlier fix2col.sty, was written by
% Frank Mittelbach and David Carlisle. This package corrects a few problems
% in the LaTeX2e kernel, the most notable of which is that in current
% LaTeX2e releases, the ordering of single and double column floats is not
% guaranteed to be preserved. Thus, an unpatched LaTeX2e can allow a
% single column figure to be placed prior to an earlier double column
% figure. The latest version and documentation can be found at:
% http://www.ctan.org/tex-archive/macros/latex/base/



%\usepackage{stfloats}
% stfloats.sty was written by Sigitas Tolusis. This package gives LaTeX2e
% the ability to do double column floats at the bottom of the page as well
% as the top. (e.g., "\begin{figure*}[!b]" is not normally possible in
% LaTeX2e). It also provides a command:
%\fnbelowfloat
% to enable the placement of footnotes below bottom floats (the standard
% LaTeX2e kernel puts them above bottom floats). This is an invasive package
% which rewrites many portions of the LaTeX2e float routines. It may not work
% with other packages that modify the LaTeX2e float routines. The latest
% version and documentation can be obtained at:
% http://www.ctan.org/tex-archive/macros/latex/contrib/sttools/
% Documentation is contained in the stfloats.sty comments as well as in the
% presfull.pdf file. Do not use the stfloats baselinefloat ability as IEEE
% does not allow \baselineskip to stretch. Authors submitting work to the
% IEEE should note that IEEE rarely uses double column equations and
% that authors should try to avoid such use. Do not be tempted to use the
% cuted.sty or midfloat.sty packages (also by Sigitas Tolusis) as IEEE does
% not format its papers in such ways.

% *** PDF, URL AND HYPERLINK PACKAGES ***
%
\usepackage{url}
% url.sty was written by Donald Arseneau. It provides better support for
% handling and breaking URLs. url.sty is already installed on most LaTeX
% systems. The latest version can be obtained at:
% http://www.ctan.org/tex-archive/macros/latex/contrib/misc/
% Read the url.sty source comments for usage information. Basically,
% \url{my_url_here}.


% *** Do not adjust lengths that control margins, column widths, etc. ***
% *** Do not use packages that alter fonts (such as pslatex).         ***
% There should be no need to do such things with IEEEtran.cls V1.6 and later.
% (Unless specifically asked to do so by the journal or conference you plan
% to submit to, of course. )


% correct bad hyphenation here
\hyphenation{op-tical net-works semi-conduc-tor}


\begin{document}
%
% paper title
% can use linebreaks \\ within to get better formatting as desired
%\title{``How is your community doing?'' Studying \emph{snapshot}s of development communities}
\title{Empirically Studying ``Snapshots'' of Software Development Social Communities}

% author names and affiliations
% use a multiple column layout for up to three different
% affiliations
%\author{\IEEEauthorblockN{Damian A. Tamburri\\ Patricia Lago\\ and Hans Van Vliet}
%\IEEEauthorblockA{Department of\\ Computer Science\\
%VU University Amsterdam, The Netherlands\\
%$[$d.a.tamburri, p.lago, j.c.van.vliet$]$@vu.nl}
%\and
%\IEEEauthorblockN{Martin von Weissenberg}
%\IEEEauthorblockA{Hanken
%School of Economics\\
%Department of Information Systems\\
%Helsinki, Finland \\
%martin.vonweissenberg@hanken.fi }}

\author{Damian A. Tamburri,~\IEEEmembership{Member,~IEEE,}
        Patricia Lago,~\IEEEmembership{Member,~IEEE,}\\
        Hans van Vliet,~\IEEEmembership{Member,~IEEE}
        and Martin von Weissenberg,~\IEEEmembership{Member,~IEEE}
        % <-this % stops a space
\IEEEcompsocitemizethanks{
\IEEEcompsocthanksitem D.A. Tamburri, P. Lago and H. van Vliet
are with the Information Management and Software Engineering group, VU University, Amsterdam,
The Netherlands.\protect\\
% note need leading \protect in front of \\ to get a newline within \thanks as
% \\ is fragile and will error, could use \hfil\break instead.
Find them at: $[$d.a.tamburri, p.lago, j.c.van.vliet$]$@vu.nl
\IEEEcompsocthanksitem M. von Weissenberg is with the Department of Information Systems, Hanken
School of Economics,Helsinki, Finland.\protect\\
Find him at: martin.vonweissenberg@hanken.fi}% <-this % stops a space
\thanks{}}

% conference papers do not typically use \thanks and this command
% is locked out in conference mode. If really needed, such as for
% the acknowledgment of grants, issue a \IEEEoverridecommandlockouts
% after \documentclass

% for over three affiliations, or if they all won't fit within the width
% of the page, use this alternative format:
% 
%\author{\IEEEauthorblockN{Michael Shell\IEEEauthorrefmark{1},
%Homer Simpson\IEEEauthorrefmark{2},
%James Kirk\IEEEauthorrefmark{3}, 
%Montgomery Scott\IEEEauthorrefmark{3} and
%Eldon Tyrell\IEEEauthorrefmark{4}}
%\IEEEauthorblockA{\IEEEauthorrefmark{1}School of Electrical and Computer Engineering\\
%Georgia Institute of Technology,
%Atlanta, Georgia 30332--0250\\ Email: see http://www.michaelshell.org/contact.html}
%\IEEEauthorblockA{\IEEEauthorrefmark{2}Twentieth Century Fox, Springfield, USA\\
%Email: homer@thesimpsons.com}
%\IEEEauthorblockA{\IEEEauthorrefmark{3}Starfleet Academy, San Francisco, California 96678-2391\\
%Telephone: (800) 555--1212, Fax: (888) 555--1212}
%\IEEEauthorblockA{\IEEEauthorrefmark{4}Tyrell Inc., 123 Replicant Street, Los Angeles, California 90210--4321}}




% use for special paper notices
%\IEEEspecialpapernotice{(Invited Paper)}




% make the title area
\maketitle


\begin{abstract}
%\boldmath
Engineering software is a social activity. Therefore, software project success is very often dependent on the welfare of the development community. This connection becomes especially delicate when multiple communities have to join forces (e.g. in global software engineering). So far however, software engineering discipline lacks systematic methods to steer development communities. In this paper we present ODeSSA a method to Outline Development Social Structures to Analyse them. We elaborated the method by conducting a real-life case-study in a large software development organisation. 
From the case-study we found that community \emph{snapshot}s can have a limited number of forms, each with their own peculiarities. In addition, we found that ODeSSA helps detecting many organisational barriers, both negative and positive ones. 
%If they are, they can be tackled by analysing the status of the software development community. We illustrate with a case-study.
%, by making its key attributes explicit. In addition, we suggest ways in which practitioners can use the community state, to adjust the relative communities (e.g. by adding specific technologies or changing organisational goals). 
\end{abstract}
% IEEEtran.cls defaults to using nonbold math in the Abstract.
% This preserves the distinction between vectors and scalars. However,
% if the conference you are submitting to favors bold math in the abstract,
% then you can use LaTeX's standard command \boldmath at the very start
% of the abstract to achieve this. Many IEEE journals/conferences frown on
% math in the abstract anyway.

% no keywords

% For peer review papers, you can put extra information on the cover
% page as needed:
% \ifCLASSOPTIONpeerreview
% \begin{center} \bfseries EDICS Category: 3-BBND \end{center}
% \fi
%
% For peerreview papers, this IEEEtran command inserts a page break and
% creates the second title. It will be ignored for other modes.
\IEEEpeerreviewmaketitle

\section{Introduction}

The craft of software is becoming more and more a social activity \cite{specissue}. Several studies suggest that engineers spend between 70 and 85\% of their time working with other people (e.g. end-user focus groups, managers, business sponsors, etc.) \cite{socialbook}. This means that in 70-85\% of times, software engineering success is a community effort, rather than a success of single individuals. This figure becomes critical with the increase of diversity and size of the development community. For example, Global Software Engineering (GSE) is a development strategy entailing global teams to collaborate together from different locations in different timezones \cite{gsdbook}. In this circumstance, distances in time, space and culture, magnify project risks connected to failures in the development social structure \cite{empirglob,nachiappan,icgseoss}.
The problem persists since software practitioners lack a systematic method to study (global) development social structures. In previous work \cite{specissue} we provided a mechanism to uncover social communities latent in software development but we learned that it shows many limitations when used in real-life scenarios.
%As a consequence, there is no way for software engineering practitioners to effectively act upon the development community they are part of.
To tackle this issue we conducted a real-life case-study in a big software development organisation. 

The study resulted in ODeSSA, a semi-automatic method for Outlining Development Social Structures to Analyse. ODeSSA allows the observers of a social structure (more simply, a community) to obtain analysable ``\emph{snapshot}s'' for it. A community \emph{\emph{snapshot}} is obtained by combining: (a) answers to a questionnaire (see Fig. \ref{question} also available online\footnote{\url{https://docs.google.com/document/d/~1lZ2IrQzd3uBqk4eivqA6RStiquzImxEETbdtCKRdHjY/edit}}); (b) visits to a decision-tree (previously introduced in \cite{specissue}). 
%
%The method uses the questionnaire in Fig. \ref{quest} to gather and organise necessary information about the observed community. Finally, the method uses the decision-tree form \cite{specissue} to match the information in the questionnaire with community types. 
 
We made three key observations: (a) ODeSSA uncovers points of indecision on the observed \emph{snapshot}, these points deserve further attention by the observer; (b) the community \emph{snapshot} captured through ODeSSA can have a limited number of possible forms, each with its own particular characteristics; (c) ODeSSA makes it possible to uncover organisational barriers that may hinder communication during software development. 
\begin{figure}[h!]
%\hspace{-.6cm}
%\begin{sideways}
\includegraphics[width=3.5in]{quest}%
%\end{sideways}
\caption{a questionnaire to snapshot development communities.}\label{question}
\end{figure}
%these forms are worthy of further study since they could act as proxies for quality of the observed community.
%In addition we suggest ways to manipulate the community state, e.g. to change community type or its focus. 
%In algorithms theory, a process state is given by the (current) values of all variables within the scope of the process \cite{algos}.
%Our method is based on empirical evidence. 

%%%%%%%%%% NOOOOOO!!!
%We found that a \emph{social community} is a specific type of social network for which certain attributes remain constantly true.

%We found that the status of a community is given by a set of paths on a community decision-tree.
%For example, imagine you are working within a UK open-source forge that collaborates with another open-source community from the US. This basic information conveys two critical informations: (a) the resulting community (US+UK) is \emph{dispersed}, i.e. not collocated; (b) the resulting community is \emph{informal}. According to \cite{ossslr,specissue} the presence of these two attributes is necessary and sufficient to identify an \emph{Informal Network} social community type.
%

%Finally, we compare the identified social community attributes (i.e. the community state) with empirical results from previous work \cite{ossslr}. This last step allows us to understand what can be done to improve the communities, e.g. by encouraging changes in attributes.
%The benefits of using our method are threefold: first, practitioners can use the questionnaire as a checklist for missing management information; second, the questionnaire can be used as a guideline to engineer communities for a specific software development problem; third, the community state we obtain can be cross-referenced with development velocity to evaluate community effectiveness in the current state.
%\begin{figure*}
%\hspace{.5cm}
%%\begin{sideways}
%\includegraphics[width=6.3in]{tree}%
%%\end{sideways}
%\caption{a decision-tree to uncover social communities \cite{specissue}}\label{tree}
%\end{figure*}
%%
%The rest of the paper is structured as follows: Section \ref{devcomm} provides some background on software engineering and the study of its social aspects; Section \ref{rq} provides our research questions; Section \ref{mm} explains materials and methods; Section \ref{cs} presents the case-study, while Section \ref{res} discusses results. Finally, Section \ref{conc} concludes the paper.

% no \IEEEPARstart
%This demo file is intended to serve as a ``starter file''
%for IEEE conference papers produced under \LaTeX\ using
%IEEEtran.cls version 1.7 and later.
%% You must have at least 2 lines in the paragraph with the drop letter
%% (should never be an issue)
%I wish you the best of success.
%\hfill mds

\subsection{Studying and Supporting Software Development Communities}\label{devcomm}
%\begin{itemize}
%\item related work from socio-technical congruence
%\item related work dana damian
%%\item conway's law?
%\item other works in ged for social studies
%\item productivity and agile methods to be adopted in gsd
%\end{itemize}

Literature in GSE already recognises the need to study global communities. More in particular, many works such as \cite{nachiappan} have studied empirically the effect of organisational structures on the quality of software. These works motivate further studies to understand, represent and support social structures and communities. In addition, similar works such as \cite{uls,rel9} study the effect of socio-technical congruence on designing global systems as well as the process of global engineering. Such works suggests that understanding how coordination requirements map to their people counterpart is critical to steer information across global communities successfully. Our work is limited to identifying communities and finding ways to study their observable characteristics. 
Other works such as \cite{Davenport2001} study the impact of successful communities on socio-technical problems (e.g. coordination, shared understanding, etc.). These works take a more social and organisational approach, pointing out the paramount importance of supporting explicitly the work of communities rather than allocating tasks to single individuals. In previous work \cite{icgseoss,ossslr}, we found that observable \emph{social communities} captured in \cite{Davenport2001} are a composition of known social community types \cite{ossslr}.
In addition, works such as \cite{scrum} study the impact of agile methods on a large scale distributed software engineering attempt. Understanding the impact and changes on global communities forced in by organisational changes such as agile adoption, is key to successful change, especially on a global scale.
Our work in this paper does not explore the impact of organisational decisions affecting the development community, but introduces a method to understand the community by \emph{snapshot}ting its current status, e.g. to support related organisational decisions such as agile adoption.
Similarly, works such as \cite{rel,rel4} investigate the impact of organizational decision-making on software development. For example, understanding if the current community \emph{snapshot} is performant (or even compatible) with offshoring is essential for the community to ``go global''.
%
%
%TODO: elaborate with the rest of related work, e.g. references on organisational change, other empirical studies in global communities or global communication \\

\subsection{Research Questions}\label{rq}

This paper reports an empirical study of software development organisations. We pay particular attention on the layout of the organisation. Our work was driven by the following research questions:

\begin{center}
{\bf What elements of a development community should be made explicit to study its current status?}
\end{center}
Previous research in organisational studies, social-networks analysis (SNA) and software engineering, suggest there are many variables at play within a software development community. In previous research \cite{ossslr,specissue}, we provided an extensive list of attributes that define communities. In this work we investigated a big, real-life software development organisation, to identify which combinations of the introduced attributes are essential to define a \emph{snapshot} for the observed organisation. By examining interviews and other sources of organisational evidence, we found a way to represent a \emph{snapshot} of an observed community, such that the \emph{snapshot} can be further analysed.

\begin{center}
{\bf What are the possible forms of the development community status, or \emph{snapshot}?}
\end{center}
In previous work \cite{icgseoss,specissue} we found many combinations of communities that can represent software development.
Studying interviews and analysing grounded-theory results from available organisational data, we found the possible forms that a community status can assume. Each of these forms manifests with own characteristics and should be acted upon accordingly.

\begin{center}
{\bf How does organisational change influence the form of a development community \emph{snapshot}?}
\end{center}
Much research is aimed at establishing the influence of organisational change on the people and process of software engineering. For example, In previous work \cite{specissue}, we found that the ``shift'' to agile methods could be an organisational change with both positive and negative impact. The organisation we studied was undergoing an organisational ``shift''. Studying the opinions of developers and technicians, and the characteristics of the community \emph{snapshot}, we found barriers that clashed with the success of organisational change.

%\begin{itemize}
%\item explain impact of the question and importance of answering
%\end{itemize}
%%%%%%%%%%Providing an intelligible shape for an observable software development community is the first step towards efficient steering of community operations. 
%%%%%%%%%%
%%%%%%%%%%For example, studying \emph{snapshot}s of global communities enables the development of ad-hoc adaptations based on current (or desired) community \emph{snapshot}s. Global software development literature acknowledged that global communication is compromised by the presence of organisational barriers within and between communities \cite{gse,gsdbook}. Given the presence of barriers in certain \emph{snapshot}s, organisations could plan organisational adaptations to those barriers.
%%%%%%%%%%
%%%%%%%%%%Moreover, studying \emph{snapshot}s of communities, managers can plot cross-community communication patterns that match exactly the current situation. For example, during a Scrum sprint, Scrum coaches can evaluate \emph{snapshot}s of the development community to understand its productivity and adapt it as needed (e.g. to avoid ``Crunch-times'' in Scrum).

%
%previous research in software engineering, organisational sciences and systems theory has observed many social community types. Over time, many attributes were observed and used to characterise these community. We analyse empirical evidence with the aim of pinpointing the attributes that are necessary and sufficient to capture the state of a certain community
%
%Second, {\bf how can we act on the status of a software development community?}  .\\
%
%Finally, {\bf how can practitioners use the state-of-the-art in social communities to improve development collaboration?}  .\\

% 
%\hfill January 11, 2007
%
%\subsection{Subsection Heading Here}
%Subsection text here.
%
%
%\subsubsection{Subsubsection Heading Here}
%Subsubsection text here.
%

\section{Materials and Methods}\label{mm}
%This paper introduces a method to study a development community using a case-study. 
The conclusions in this paper are drawn from a real-life industrial case-study. This section explains what materials we used to carry out the case study and how we examined such materials.

\subsection{Materials}
%TODO: explain that two people conducted the case-study with the same material in parallel with minimal interaction on the case study, as overseen by two senior researchers. when it was finished the results were compared for evaluation. the method can be generalised since both researchers followed the same emergent path and encountered the same probs, that were solved in pretty much the same way.\\
%The following sections describe materials and methods in more detail.

%\begin{itemize}
%\item describe work in the SLR
%\item describe work in ICGSE paper
%\item describe work in IEEE Sw paper
%\item related work
%\end{itemize}
%
%\subsection{Materials Used}
The empirical evidence used in our work, stems from empirical research conducted in a big organisation, active in the mobile technologies market. The organisation (called Company X from now on) is multinational that develops both hardware and software components for end-users' mobile-phones and apps to enrich user experience. Company X employs over 130K people in 16 worldwide sites, with sales in over 160 countries. At the time the empirical research was conducted (early 2012), the organisation had a net market sale of 42.4 billion euros. 

Our empirical data describes the overall organisation and some of its software development sites. The data also reports on the collaboration between Company X's Scrum development community and a constellation of open-source communities. 

Three key sources of evidence were used:
\begin{enumerate}
\item The first key resource (called REF1 from now on) is a research report studying agile and open-source adoption in Company X. The study was conducted using grounded-theory \cite{gt}. The study features six interviews with focus groups from Company X. Focus groups had varied expertise, from areas such as project management to agile coaches to code developers. The focus sessions are analysed through grounded theory. The study offers an overview of the organisational structure of Company X as a whole (i.e. as a global community) as well as explaining its local sites. The research focuses on a division of Company X, working on a specific project release, at the time of investigation. The report is 112 pages long, we use page numbers for the purpose of traceability. 

\item The second key resource (called REF2 from now on) is a research report studying three software engineering teams within Company X. The study features 16 interviews harvested through accidental and snowball sampling \cite{Goodman}. Interviewees had varied backgrounds but an average work experience in Company X of about 10 years. The sample consisted of both sexes, however, with a minority of females. As a result, the organisational details of this report focus around software process, software coding methods, development best-practices adopted, problems found, and ultimately, the people's perspective on all the above. The report is 97 pages long, we use page numbers for the purpose of traceability.

\item The third key resource (called AGILE SLIDES from now on) is a set of tutorial slides that agile coaches within Company X used to introduce X's organisational structure/process to newly appointed staff members. The slides provide an overview of the agile process and open-source integration protocols in place within Company X at the time of investigation. The report is 4 pages long, we use page numbers for the purpose of traceability.
%\item explain unique input by martin
\end{enumerate}

All of the above sources are dated around 2012 and are protected by non-disclosure agreements \footnote{available upon signed request}. At the time of investigation, Company X was going through many organisational changes and both studies aimed at understanding the impact of organisational change on teams, their production velocity, community productivity and end-product quality. 

\subsection{Methods}

We conducted two parallel case-studies, following guidelines in \cite{runeson}. Given the same input materials, two parallel case-studies were started, with the same objective, i.e. to evaluate real-life development communities from a big industrial case using an application of the decision-tree theory from \cite{specissue}. 
%
%The study was intended to validate and enrich the method from our previous work, to make it usable in real-life practice. 
The connotation of our case-study was both exploratory and descriptive. We set out to find an intelligible form for the observed development community, such that the outlined form could be further studied, e.g. compared to organisational metrics for Company X.

The parallel study was used to assess the generalizability of the approach. In addition, the use of a parallel study was meant to minimise bias and unreliability of results through observer triangulation, according to \cite{stake}. While the two studies were started at the same time, both researchers conducting them were prohibited to share increments of partial results. Weekly meetings were planned to assess progress status only. The supervision of two senior researchers was used throughout the duration of the study.

Moreover, data sources triangulation \cite{stake} was used to aid the generalizability of our results. To this purpose, three different sets of data were used. Two of these were obtained through surveys and other empirical enquiry methods.

%using questionnaire to gather and organise organisational information needed plus the decision-tree to match with visits and comm. types use the piece below as a start\\
%
%

\section{Results}\label{cs}

To apply the method in \cite{specissue} on our case-study, we analysed the evidence available. 
%The method allows associating a single community type to multiple observable attributes, through a decision-tree (see Fig. \ref{tree}). 

We noticed immediately that the method in \cite{specissue} works best to identify a single observable community. This underlying limitation clashed with the sheer size and cluttering of organisational information from our big real-life scenario. We needed a mechanism to sort out the evidence available. To this purpose, we used the questionnaire in Fig. \ref{question}. The questionnaire gathers necessary organisational information, without limiting the observer's to a single path on the decision-tree from \cite{specissue}. Column 1 of the questionnaire provides the question that needs answering; Column 4 provides an ID for the question, to ease the mapping on the decision-tree from \cite{specissue}; finally, Column 5 requires the observer to capture the rationale of the answer, i.e. the piece of organisational evidence that sustains the decision.

The evidence available reflected the questionnaire in Table \ref{answers}. To remain simple, we present only one of the pieces of evidence that led us to answer each question. This piece of evidence is reported in column 5 of Table \ref{answers}. The following description summarises the details from Table \ref{answers}:
\begin{center}
\emph{Company X works with an integrated Scrum model, using many divisions of work, e.g. business units, or product teams. Communication takes place informally only during certain phases. Other phases (such as stakeholder reviews, corporate progress meetings, product release plans, etc.) are regulated by rigid governance policies and protocols. New personnel is formally evaluated and selected to integrate the community. Both Company X and its clients actively encourage and motivate the development community to increase engagement and cohesion. Organisational culture (such as best-practices, recurrent product requirements, collaboration patterns, etc.) is neither collected nor maintained. Rather, the experience of personnel is established during evaluation of background and skills (i.e. during personnel review and initial evaluation before employment). Finally, Company X doesn't employ any mechanism to visualise product development process but leaves it up to Scrum masters and other leadership to disseminate results and increase their visibility.}
\end{center}
%%%%%% Martin!!! please revise/integrate!!! :)

We were able to clearly answer most questions except question 3 and 11.

For question 3 ``do all community members exhibit a different location?'', we found organisational information leading to both a ``YES'' or a ``NO''. Moreover, we noticed that organisational information leading to ``YES'' was related to Company X as a whole. Conversely, organisational information leading to ``NO'' was referring to only a few sites within Company X. 

For question 11 ``do community members require formal status evaluation?'', it was impossible for us to answer, given the lack of available information.
%TODO: put figure of the answered questionnaire. on the figure, the pivot answers need to be with interrogation marks\\
%TODO: explain the answers\\
\begin{table*}
%\hspace{-.6cm}
%\begin{sideways}
%%\includegraphics[width=7.5in]{answers_new}%
%\end{sideways}
\begin{tabular}{|>{\raggedright}p{5cm}|c|c|c|>{\raggedright}p{10.2cm}|}
\hline 
\textbf{QUESTION} & \textbf{Y} & \textbf{N} & \textbf{ID} & \textbf{RATIONALE}\tabularnewline
\hline 
Can I distinguish a non-trivial structure within the observed people? & x &  & 1 & REF 2 and3 point out to a non-trivial structure in which the professionals
within Company X are organized. There are divisions, products, innovation
areas, etc.\tabularnewline
\hline 
Do community members share a situated practice? &  & x & 2 & Page 68 of REF 2, authors stressed that some people worked off-site
on the same product. This is repeated multiple times in other parts
of the research report.\tabularnewline
\hline 
Do all community members exhibit a different location? & \multicolumn{1}{c}{} & ? & 3 & in page 68 REF 2 suggested that a technology for localization and
involvement of people from off-site workers was needed but not explicitly
present in a technological form. It follows that some components of
the organisational structure were indeed decentralized. However, other
both REF 2 and 3 speak about a collocated development process.\tabularnewline
\hline 
Do community members use informal communication only? &  & x & 4 & Rigid selection and governance protocols are in action to request
additional people to work on certain tasks (page 84 of REF 2 + AGILE
SLIDES).\tabularnewline
\hline 
Do community members exhibit longevity bound to a project? & x &  & 5 & Page 2 of AGILE SLIDES showed the strategic portfolio duration. This
is fixed to increments of 2-5 years.\tabularnewline
\hline 
Do community members actively try to increase visibility of their
operations and results?  &  & x & 6 & REF 1, page 59 stated that all production items and Scrum backlogs
as well as all operations are explicitly planned to be visible and
transparent to all working to a product. Therefore visibility is actively
pursued NOT by members but by hand of their organizational sponsor.\tabularnewline
\hline 
Do community members actively pursue cohesion?  & x &  & 7 & In page 30 and 65, REF 1 noted that anywhere in the organizational
structure, the intervention of senior management was needed and present
to oversee end-product quality, contract value and people encouragement/bonding.
This is consistent with increasing the cohesion of Scrum teams.\tabularnewline
\hline 
Are community members expected to demonstrate ROI to business sponsors?  & x &  & 8 & In page 64 and subsequent, REF 2 stated that demos phases were acting
as checkpoints for customers to evaluate ROI.\tabularnewline
\hline 
Does the community effectiveness depend on people\textquoteright{}
engagement?  &  & x & 9 & In page 74, REF 2 reported that an interviewee had a reduced engagement
in the project, caused by a reduced set of roles and responsibilities.
This however, did not reduce productivity and consequently did not
lower community effectiveness (page 64 and following).\tabularnewline
\hline 
Do community members actively accumulate and foster organisational
culture?  &  & x & 10 & there was no indication that business units or organisation were actively
collecting and managing organizational culture (e.g. best-practices,
usage scenarios, wicked problems etc.).\tabularnewline
\hline 
Do community members require formal status evaluation? & \multicolumn{1}{c}{} & {*} & 11 & Not enough information.\tabularnewline
\hline 
Do community members tackle one explicit organisational problem?  & x &  & 12 & In page 24 of REF 1 authors stress that there is a clear-cut definition
of the mission/problem statement both of the organisation and its
business units. This constrains the operativeness of the community
to achieving the prescribed directives.\tabularnewline
\hline 
Do community members endure formal governance practices?  &  & x & 13 & REF 1 stated in page 30 and subsequent that teams are self-managing,
self-organizing, cross-functional and responsible. This means that
there are no fixed and formally agreed governance guidelines that
arrange and constrain the work of teams.\tabularnewline
\hline 
\end{tabular}
\caption{Company X, questionnaire with answers.}\label{answers}
\end{table*}
To evaluate results we mapped the answers in Figure \ref{answers} with the decision-tree from \cite{specissue}. The result of the mapping to the decision-tree is shown in Figure \ref{result}. 
%We noticed that, when mapped to the decision-tree, the point of indecision marks the start of another path on the tree, i.e. another community. 
%The point of indecision is a \emph{pivot} decision-tree visits.
%TODO: then proceed with showing the tree visit.\\

Finally, we concluded the case-study by applying the decision-tree from previous work \cite{specissue}. This allowed us to understand the communities emerging within Company X's organisational structure. Figure \emph{result} shows results of this application. Two social communities were identified, namely, Networks of Practice (NoPs - part B of Fig. \ref{result}) and Workgroups (WGs - part A of Fig. \ref{result}). 

According to our data, Company X is organised as a globally dispersed (i.e. separated by distance in time and space) set of sites. No collocated work sessions are planned or supported in any one site. Rather, work stays dispersed and relies on dedicated digital means for sharing and synchronisation. The NoP is self-organising and open to collaboration with other communities. In our case, for example, some (parts of) sites, were explicitly collaborating with (and connected to) open-source communities.

Moreover, sites within Company X, are organised as tight Workgroups. Internal cohesion and enthusiastic collaboration are key characteristics. Frequent collocated Workgroup meetings are used to verify productivity. Workgroups in Company X have a wide business agenda and are focused on delivering valuable products rather than delivering time-bound projects. 

We found that the point of indecision, i.e. the DISPERSION node (part C of Fig. \ref{result}) acts as a \emph{pivot} between the two identified community types. The attributes leading to the identification of a NoP were referred to global details of Company X (DISPERSION set to YES). Conversely the attributes leading to the identification of a WG are referred to localised divisions or business units within Company X (DISPERSION set to NO). 

\begin{figure*}
\hspace{-.2cm}
%\begin{sideways}
\includegraphics[width=7in]{results}%
%\end{sideways}
\caption{Company X, a case-study: decision-tree visits.}\label{result}
\end{figure*}

\subsection{Replication}

After our analysis was done, we explored the generality of our approach and findings, by comparing results from both parallel studies. We found that both efforts were carried out with the same underlying approach. We concluded that the approach can be generalised into ODeSSA, a method to outline development social structures to analyse them by capturing \emph{snapshots}. The ODeSSA method is organised as follows:
\begin{enumerate}
\item \emph{Gather Data:} The questionnaire in Fig. \ref{question} must used as a checklist to gather necessary organisational details. Once analysed, these details must enable answering the questions in the questionnaire. Missing data can be gathered by directly surveying (e.g. with online forms) members of the community under observation. Alternatively, other quantitative mechanisms can be used, such as Social-Network Analysis (SNA) \cite{Rosso2009}. 
\item \emph{Fill-in Questionnaire:} The questionnaire must then be filled out. The questionnaire is a lightweight mechanism to analysing observable development communities, rather than using complex coding schemes on available evidence, as seen previously in literature \cite{Hustad2010,Rosso2009}.
\item \emph{Identify Pivots:} Pivot points are points of indecision within the questionnaire. These points deserve further attention since the indicate one of two things: (a) there is a conflict or inconsistency in the analysed snapshot; (b) there is a sub-community within the analysed community.
\item \emph{Map to Decision-Tree:} Once organisational evidence is ready, the decision-tree can be mapped on the questionnaire to uncover community types. A community \emph{snapshot} can be obtained by combining information from the questionnaire with reflected community types.
\item \emph{Evaluate snapshot form:} The properties of the obtained \emph{snapshot} must be evaluated. We found that the snapshot can present a limited number of forms (see Section \ref{res} for additional details).
\end{enumerate}

At this point, considerations on the snapshot can be associated with quantitative metrics or numbers concerning the company's current organisational performance. This can be useful in both academia and industry. For academic research, it could be useful to find patterns and causal relations between metrics for organisational performance and community snapshots, e.g. to establish the feasibility of certain community layouts for certain organisational goals. For practitioners, this can be useful to reflect on ways in which the current organisational layout could be steered or extended to improve its performance.
%\begin{itemize}
%\item explain how the case-study was conducted
%\item explain what the evidence used to give an answer to the questionnaire
%\item show the questionnaire and the visits
%\item explain types found in the case
%\item explain the design of the case-study: (a) through a questionnaire\footnote{link to URL of ODeSSATemplate.gdoc}; (b) the decision-tree described in \cite{specissue});
%\item explain the way we had to proceed in separate cases and weekly meet to share ideas and discussion of approach only limited to clarification of tasks
%\item overseeing of two senior researchers
%\end{itemize}

% An example of a floating figure using the graphicx package.
% Note that \label must occur AFTER (or within) \caption.
% For figures, \caption should occur after the \includegraphics.
% Note that IEEEtran v1.7 and later has special internal code that
% is designed to preserve the operation of \label within \caption
% even when the captionsoff option is in effect. However, because
% of issues like this, it may be the safest practice to put all your
% \label just after \caption rather than within \caption{}.
%
% Reminder: the "draftcls" or "draftclsnofoot", not "draft", class
% option should be used if it is desired that the figures are to be
% displayed while in draft mode.
%
%\begin{figure}[!t]
%\centering
%\includegraphics[width=2.5in]{myfigure}
% where an .eps filename suffix will be assumed under latex, 
% and a .pdf suffix will be assumed for pdflatex; or what has been declared
% via \DeclareGraphicsExtensions.
%\caption{Simulation Results}
%\label{fig_sim}
%\end{figure}

% Note that IEEE typically puts floats only at the top, even when this
% results in a large percentage of a column being occupied by floats.

\section{Discussion and Findings}\label{res}

We analysed our results and made three key observations. The following text states the observation and provides an interpretation in \emph{italic}.

%First, the method used during the case-study can be generalised. Second, we found that the questionnaire has many uses other than gathering the needed organisational information. Second, we found that \emph{pivot} points can have multiple answers for the same community, this detail makes them worthy of particular attention. Third, we found that data in the questionnaire can assume five possible forms, each with some interesting properties. Fourth, organisational barriers can either be useful to shape an organisation or harmful to its operations. The text below explains all findings in more detail. In \emph{italic} we provide a discussion of the finding.

%\subsection{ODeSSA: a method for Outlining Development Social Structures to Analyse}
%
%diversi usi della tabella:
%1. social communities multiple
%2. incompletezza? fungi da checklist
%3. ingegnerizzate una community per sw dev in forward
% inoltre, storia dei pivot e cosa significano
% inoltre, possibili forme e cosa significano

%\subsection{Findings and Observations}

%{\bf First, we found that observable organisational information misleads the detection of communities.}
%%
%\emph{This suggests that observable organisational information is overloaded with data. A questionnaire is an essential piece to organise and contain only the necessary and sufficient information to uncover a community type. In addition, this finding also suggests that some organisational information might be missing or incomplete. The questionnaire can also be used as a checklist to make sure all needed information is actually available. Finally, this finding implies that our method for community detection cannot be fully automated.}
%TODO: make examples of each of the above statements.\\


%A community \emph{\emph{snapshot}} can be obtained by combining two key informations: (a) answers to the questionnaire in Fig. \ref{quest}; (b) visits to the decision-tree from \cite{specissue}. 
%
%{\bf First, the questionnaire part of the ODeSSA method� }
%TODO: finish explaining\\
%TODO: add example\\
%TODO: explain how the finding was extracted from our case study and how and why it can be generalized\\

First, pivot points represent key variation points distinguishing a community from its subcomponents. For example, In our case study we found evidence that led to discovering NoPs. Likewise, we discovered evidence that could lead to discovering WGs. The only point of indecision was resting in the DISPERSION node (see part C of Fig. \ref{result}). Referencing both visits with the evidence available we observed that the path leading to WGs was only visitable if we focused on some particular sites within Company X. Conversely, the path leading to NoPs was only visitable if we considered Company X as a whole, without focusing on any single site or unit within. It follows that the DISPERSION node was a key organisational variation point between Company X as a whole and its various sites.
%More in particular you need to visit the tree once per every possible combination of all \emph{pivots}.
\emph{Pivots can be used as drivers for the analysis of the community snapshot. Their presence indicates that there is misalignment between a community as a whole and (some of) its components. In addition, the presence of pivots in our big industry case-study, suggests that presence is connected to organisational complexity. This indicates that pivots are worthy starting-points to analyse the community snapshot.}

%For example, suppose your organisational information suggests the presence of both a Formal Network (FN) and a Network-of-Practice (NoP). The pivot-point in this case is ``formality''. It follows that the community you are observing behaves formally at one level (for example on a global scale, or between sites) and informally at another level (for example between divisions or business units). It's at least something you might want to think about, if the community you are observing exhibits communication problems. Moreover, suppose you are investigating communities that should all behave in the same way. The pivot-points suggest who is diverging. You might want to consider if and why the divergence is a good thing.}
%TODO: use example of pivot from case-study\\

Second, we found that the information in the ODeSSA questionnaire can assume five possible forms: (1) \emph{simple} status, represents the presence of a single path; (2) \emph{enriched-community} status, represents the presence of a path blended with attributes from other types; (3) \emph{complex} status, represents the presence of a more than one path; (4) \emph{uncertain} status, represents the presence of sparse attributes that don't represent any path but only broken segments.
\emph{The five types can be used as a metric to evaluate community complexity. If the community matches a single path then it's more controllable, since it refers to a single type known to literature. Steering and management complexity increases the more the type diverges from simple.}
TODO: REVISE!!! finish explaining the fifth status and elaborate on the others; make sure that they are all\\
TODO: add example\\

Finally, organisational literature defines barriers as mechanisms or circumstances that filter organisational activity \cite{Correia2010}. We found that many barriers are present in software development communities but are not necessarily a bad thing.\\
TODO: needs the reasoning about barriers Me+Hans \\
TODO: Add example\\
%If they are, then they constrain people's communication. 
%\section{Discussion}\label{disc}

% An example of a double column floating figure using two subfigures.
% (The subfig.sty package must be loaded for this to work.)
% The subfigure \label commands are set within each subfloat command, the
% \label for the overall figure must come after \caption.
% \hfil must be used as a separator to get equal spacing.
% The subfigure.sty package works much the same way, except \subfigure is
% used instead of \subfloat.
%
%\begin{figure*}[!t]
%\centerline{\subfloat[Case I]\includegraphics[width=2.5in]{subfigcase1}%
%\label{fig_first_case}}
%\hfil
%\subfloat[Case II]{\includegraphics[width=2.5in]{subfigcase2}%
%\label{fig_second_case}}}
%\caption{Simulation results}
%\label{fig_sim}
%\end{figure*}
%
% Note that often IEEE papers with subfigures do not employ subfigure
% captions (using the optional argument to \subfloat), but instead will
% reference/describe all of them (a), (b), etc., within the main caption.


% An example of a floating table. Note that, for IEEE style tables, the 
% \caption command should come BEFORE the table. Table text will default to
% \footnotesize as IEEE normally uses this smaller font for tables.
% The \label must come after \caption as always.
%
%\begin{table}[!t]
%% increase table row spacing, adjust to taste
%\renewcommand{\arraystretch}{1.3}
% if using array.sty, it might be a good idea to tweak the value of
% \extrarowheight as needed to properly center the text within the cells
%\caption{An Example of a Table}
%\label{table_example}
%\centering
%% Some packages, such as MDW tools, offer better commands for making tables
%% than the plain LaTeX2e tabular which is used here.
%\begin{tabular}{|c||c|}
%\hline
%One & Two\\
%\hline
%Three & Four\\
%\hline
%\end{tabular}
%\end{table}


% Note that IEEE does not put floats in the very first column - or typically
% anywhere on the first page for that matter. Also, in-text middle ("here")
% positioning is not used. Most IEEE journals/conferences use top floats
% exclusively. Note that, LaTeX2e, unlike IEEE journals/conferences, places
% footnotes above bottom floats. This can be corrected via the \fnbelowfloat
% command of the stfloats package.

\section{Threats to Validity}
\begin{itemize}
\item Minna and Ossi are two different situations in two different points of Company X and different levels of abstraction. results confirmed by executing the case-study twice, contemporarily by two different researchers.
\item observations are made on only a single case-study.
\end{itemize}

\section{Conclusion and Future Work}\label{conc}

TODO: explain possible followup studies, connected to the notion of community \emph{snapshot}.\\
% conference papers do not normally have an appendix

% use section* for acknowledgement
\section*{Acknowledgment}

The authors would like to thank beer, Mrs. Mirella Sangiovanni's home-made sweets and her mother's home-made cheese.

% trigger a \newpage just before the given reference
% number - used to balance the columns on the last page
% adjust value as needed - may need to be readjusted if
% the document is modified later
%\IEEEtriggeratref{8}
% The "triggered" command can be changed if desired:
%\IEEEtriggercmd{\enlargethispage{-5in}}
% references section

% can use a bibliography generated by BibTeX as a .bbl file
% BibTeX documentation can be easily obtained at:
% http://www.ctan.org/tex-archive/biblio/bibtex/contrib/doc/
% The IEEEtran BibTeX style support page is at:
% http://www.michaelshell.org/tex/ieeetran/bibtex/
\bibliographystyle{IEEEtran}
% argument is your BibTeX string definitions and bibliography database(s)
\bibliography{community}
%
% <OR> manually copy in the resultant .bbl file
% set second argument of \begin to the number of references
% (used to reserve space for the reference number labels box)
%\begin{thebibliography}{1}
%
%\bibitem{IEEEhowto:kopka}
%H.~Kopka and P.~W. Daly, \emph{A Guide to \LaTeX}, 3rd~ed.\hskip 1em plus
%  0.5em minus 0.4em\relax Harlow, England: Addison-Wesley, 1999.
%
%\end{thebibliography}




% that's all folks
\end{document}


